\documentclass[margin,line]{res}


\oddsidemargin -.5in
\evensidemargin -.5in
\textwidth=6.0in
\itemsep=0in
\parsep=0in
% if using pdflatex:
%\setlength{\pdfpagewidth}{\paperwidth}
%\setlength{\pdfpageheight}{\paperheight} 
\newcommand{\ts}{\textsuperscript}

\newenvironment{list1}{
  \begin{list}{\ding{113}}{%
      \setlength{\itemsep}{0in}
      \setlength{\parsep}{0in} \setlength{\parskip}{0in}
      \setlength{\topsep}{0in} \setlength{\partopsep}{0in} 
      \setlength{\leftmargin}{0.17in}}}{\end{list}}
\newenvironment{list2}{
  \begin{list}{$\bullet$}{%
      \setlength{\itemsep}{0in}
      \setlength{\parsep}{0in} \setlength{\parskip}{0in}
      \setlength{\topsep}{0in} \setlength{\partopsep}{0in} 
      \setlength{\leftmargin}{0.2in}}}{\end{list}}


\begin{document}

\name{Andrew M. Kent \vspace*{.1in}}

\begin{resume}
\section{\sc Contact Information}
\vspace{.05in}
\begin{tabular}{@{}p{2in}p{4in}}
Lindley Hall 215           & {\it E-mail:} andmkent@iu.edu \\
Indiana University         & {\it Fax:}  (812) 855-4829 \\
150 S. Woodlawn Ave.       & {\it Github:} github.com/pnwamk  \\       
Bloomington, IN 47405      & {\it WWW:} pnwamk.github.io \\     
\end{tabular}


\section{\sc Research Interests}
I am interested in developing programming language based techniques
that help developers design and build robust software in real world
settings.  In particular, I am interested in making advanced
type-theoretic verification and design techniques more accessible to
developers of all levels.

\section{\sc Skills}
Research in programming language theory and formal logic, functional
and imperative programming, experience with the Unix environment and a
variety of programming languages/tools (Racket, Coq, C/C++, Agda,
Java, Python, git, etc).

\section{\sc Education}
{\bf Indiana University}, Bloomington, Indiana USA\\
%{\em Department of Statistics} 
\vspace*{-.1in}
\begin{list1}
\item[] Ph.D. Candidate, Computer Science, Ongoing
\begin{list2}
%\item Dissertation Topic:  ``TODO'' 
\item[] Advisor:  Sam Tobin-Hochstadt
\end{list2}
\vspace*{.05in}
\item[] M.S., Computer Science,  May 2017
\end{list1}

{\bf Brigham Young University}, Provo, Utah USA\\
\vspace*{-.1in}
\begin{list1}
\item[] B.S., Computer Science, graduated \emph{magna cum laude}, August 2013
\end{list1}

\section{\sc Publications}



S. Tobin-Hochstadt, M. Felleisen, R.B. Findler, M. Flatt, B. Greenman,
A.M. Kent, V. St-Amour, T.S. Strickland, A. Takikawa. \emph{Migratory
  Typing: Ten Years Later}. {\em Proc. 2\ts{nd} Summit on Advances in
  Programming Languages} (SNAPL 2017).

A.M. Kent, D. Kempe, S. Tobin-Hochstadt. \emph{Occurrence Typing
  Modulo Theories}. {\em Proc. 37\ts{th} ACM Conference on Programming
  Language Design and Implementation} (PLDI 2016). Included
  successfully evaluated artifact.

M.M. Vitousek, A.M. Kent, J.G. Siek, J. Baker. \emph{Design and
  Evaluation of Gradual Typing for Python}. {\em Proc.  10\ts{th} ACM
    Symposium on Dynamic Languages} (DLS 2014).

D.J. Kennard, A.M. Kent, W.A. Barret. \emph{Linking the Past:
  Discovering Historical Social Networks from Documents and Linking to
  a Genealogical Database}. {\em Proc. 1\ts{st} Workshop on Historical
  Document Imaging and Processing} (HIP 2011).

%\vspace*{-2.5mm}
%NSF Vertical Integration of Research and Education in Statistics and
%Mathematical Sciences\\ (VIGRE) teaching fellowship.
%

\section{\sc Academic Experience}
{\bf Indiana University}, Bloomington, Indiana USA

\vspace{-.3cm}
{\em Graduate Research Assistant} \hfill {\bf May 2014 -- Present}\\
Investigating type-based program verification, evaluating gradual
typing applications in mainstream languages, and developing techniques
to bring dependent types to dynamically typed languages. Advised by
Sam Tobin-Hochstadt.

{\em Assistant Instructor} \hfill {\bf January 2016 -- Present}\\ {\em
  CSCI-B 522 and CSCI-B 401}\\ Assisted with instruction and grading for a
graduate level programming language theory course and an undergraduate
level course on the theory of computation.

\hfill \\

{\bf Brigham Young University}, Provo, Utah USA

\vspace{-.3cm}

{\em Graduate Research Assistant} \hfill {\bf August 2013 -- April
  2014}\\ Investigated the formalization of security protocol analysis
techniques (Strand Spaces) utilizing the Coq proof assistant to create
a verified basis for accessible, automated protocol analysis
techniques. Advised by Jay McCarthy.

{\em Undergraduate Research Assistant} \hfill {\bf May -- September
  2011}\\ Developed method for automatically generating historical
social networks from source documents to aid historical
research. Advised by William Berret and Tom Sederberg.

\section{\sc Talks}
Practical Dependently Typed Racket, RacketCon 2015, St. Louis, MO, USA.

Adding Practical Dependent Types to Typed Racket, Scripts to Programs Workshop 2015, Prague, Czech Republic.

\section{\sc Professional Experience}
{\bf Microsoft Research Ltd.}, Cambridge, UK

\vspace{-.3cm}
{\em Research Intern} \hfill {\bf May -- July 2017}\\
Developed and prototyped unique solutions to trusted computing problems in the cloud.


{\bf Microsoft Corp.}, Redmond, Washington USA

\vspace{-.3cm}
{\em Software Development Engineer Intern} \hfill {\bf May -- August 2012}\\
Explored optimizations and improvements for Microsoft OneNote during a
summer internship.

{\bf United States Marine Corps}, Camp Pendleton, California USA

\vspace{-.3cm}
{\em Signals Intelligence Analyst, Sergeant} \hfill {\bf November 2005 -- August 2010}\\
Provided detailed signals intelligence analysis and reporting in
support combat operations in the Al Anbar province of Iraq during two
separate deployments.

\section{\sc Open Source Involvement}

{\em Racket and Typed Racket} \hfill {\bf 2014 --
  Present}\\ Contributions to the Racket programming language,
especially Typed Racket.

\section{\sc Community}

{\em Cub Master} \hfill {\bf December 2014 -- Present}\\
Help organize combined scouting activities for youth ages 8-11.

\vspace{-.3cm} {\em Interfaith Winter Shelter Volunteer} \hfill {\bf
  January 2016 -- Present }\\ Evening shift volunteer at a low-barrier
winter homeless shelter.


\end{resume}
\end{document}
