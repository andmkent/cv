\documentclass[margin,line]{res}


\oddsidemargin -.5in
\evensidemargin -.5in
\textwidth=6.0in
\itemsep=0in
\parsep=0in
% if using pdflatex:
%\setlength{\pdfpagewidth}{\paperwidth}
%\setlength{\pdfpageheight}{\paperheight} 
\newcommand{\ts}{\textsuperscript}

\newenvironment{list1}{
  \begin{list}{\ding{113}}{%
      \setlength{\itemsep}{0in}
      \setlength{\parsep}{0in} \setlength{\parskip}{0in}
      \setlength{\topsep}{0in} \setlength{\partopsep}{0in} 
      \setlength{\leftmargin}{0.17in}}}{\end{list}}
\newenvironment{list2}{
  \begin{list}{$\bullet$}{%
      \setlength{\itemsep}{0in}
      \setlength{\parsep}{0in} \setlength{\parskip}{0in}
      \setlength{\topsep}{0in} \setlength{\partopsep}{0in} 
      \setlength{\leftmargin}{0.2in}}}{\end{list}}


\begin{document}

\name{Andrew M. Kent \vspace*{.1in}}

\begin{resume}
\section{\sc Contact Information}
\vspace{.05in}
\begin{tabular}{@{}p{2in}p{4in}}
Lindley Hall 215            & {\it Fax:}  (812) 855-4829 \\            
Indiana University         & {\it E-mail:} andmkent@iu.edu    \\         
150 S. Woodlawn Ave.   & {\it Github:} github.com/pnwamk  \\       
Bloomington, IN 47405 & {\it WWW:} pnwamk.github.io \\     
\end{tabular}


\section{\sc Research Interests}
I'm interested in developing programming language based techniques
that help developers design and build robust software in real world
settings.  In particular I hope to increase the accessibility and
effectiveness of verification techniques by exploring their gradual
application.

\section{\sc Skills}
Research in programming language theory and formal logic, functional
and imperative programming, familiarity with the Unix environment and
a variety of programming languages/tools (Racket, Coq, C/C++, Agda,
Java, Python, git, etc).

\section{\sc Education}
{\bf Indiana University}, Bloomington, Indiana USA\\
%{\em Department of Statistics} 
\vspace*{-.1in}
\begin{list1}
\item[] Ph.D. Student, Computer Science, May 2014 - Present
\begin{list2}
\vspace*{.05in}
%\item Dissertation Topic:  ``TODO'' 
\item Advisor:  Sam Tobin-Hochstadt
\end{list2}
%\vspace*{.05in}
%\item[] M.S., Statistics,  May 2000
\end{list1}

{\bf Brigham Young University}, Provo, Utah USA\\
%{\em Department of Mathematics and Statistics} 
\vspace*{-.1in}
\begin{list1}
\item[] B.S., Computer Science, August 2013
\end{list1}

\section{\sc Publications}

Occurrence Typing Modulo Theories, with David Kempe and Sam
Tobin-Hochstadt, {\em Proc. 37\ts{th} ACM Conference on Programming 
Language Design and Implementation} (PLDI), 2016.

Design and Evaluation of Gradual Typing in Python, with Michael
M. Vitousek, Jeremy G. Siek, and Jim Baker, {\em Proc.  10\ts{th} ACM
  Symposium on Dynamic Languages} (DLS), 2014.

Linking the Past: Discovering Historical Social Networks from
Documents and Linking to a Genealogical Database, with Douglas
J. Kennard and William A. Barrett, {\em Proc. 1\ts{st} Workshop on
  Historical Document Imaging and Processing} (HIP), 2011.

\section{\sc Honors and Awards} 
Graduated \emph{magna cum laude} from Brigham Young University.

Awarded NASA Space Grant Consortium Fellowship (Fall 2013 -- Winter
2014)

Top graduate in multi-service military Intermediate Communications
Signals Analysis course (Corry Station Naval Technical Training
Center, Florida, April -- August 2007)

%\vspace*{-2.5mm}
%NSF Vertical Integration of Research and Education in Statistics and
%Mathematical Sciences\\ (VIGRE) teaching fellowship.
%

\section{\sc Academic Experience}
{\bf Indiana University}, Bloomington, Indiana USA

\vspace{-.3cm}
{\em Graduate Research Assistant} \hfill {\bf May 2014 -- Present}\\
Investigating type-based program verification, evaluating gradual
typing applications in mainstream languages, and developing techniques
to bring dependent types to dynamically typed languages. Advised by
Sam Tobin-Hochstadt.

{\em Assistant Instructor} \hfill {\bf January 2016 -- Present}\\ {\em
  Various courses}\\ Assisted with instruction and grading for a
graduate level programming language theory course and an undergraduate
level course on the theory of computation.

{\bf Brigham Young University}, Provo, Utah USA

\vspace{-.3cm}

{\em Graduate Research Assistant} \hfill {\bf August 2013 -- April
  2014}\\ Investigated the formalization of security protocol analysis
techniques (Strand Spaces) utilizing the Coq proof assistant to create
a verified basis for accessible, automated protocol analysis
techniques. Advised by Jay McCarthy.

{\em Teaching Assistant} \hfill {\bf August 2013 -- April 2014}\\ {\em
  CS-330 Concepts of Programming Languages}\\ Held regular office hours
and assisted in teaching programming language concepts.


{\em Undergraduate Research Assistant} \hfill {\bf May -- September
  2011}\\ Developed method for automatically generating historical
social networks from source documents to aid historical
research. Advised by William Berret and Tom Sederberg.

\section{\sc Talks}
Practical Dependently Typed Racket, RacketCon 2015, St. Louis, MO, USA.

Adding Practical Dependent Types to Typed Racket, Scripts to Programs Workshop 2015, Prague, Czech Republic.

\section{\sc Professional Experience}
{\bf Microsoft Corporation}, Redmond, Washington USA

\vspace{-.3cm}
{\em Software Development Engineer Intern} \hfill {\bf May 2012 -- August 2012}\\
Explored optimizations and improvements for Microsoft OneNote during a
summer internship, receiving a full-time offer for employment upon completion.

{\bf United States Marine Corps}, Camp Pendleton, California USA

\vspace{-.3cm}
{\em Signals Intelligence Analyst} \hfill {\bf November 2005 -- August 2010}\\
Provided detailed signals intelligence analysis and reporting in
support combat operations in the Al Anbar province of Iraq during two
separate deployments. Additionally, trained and led a team of six
signals intelligence analysts during the second deployment to Iraq.

\section{\sc Open Source Involvement}

{\em Racket and Typed Racket} \hfill {\bf 2014 -- Present}\\ I
contribute to the Racket programming language and am a primary
contributer to the Typed Racket language.

\section{\sc Community}

{\em Cub Master} \hfill {\bf December 2014 -- Present}\\
Help organize combined scouting activities for youth ages 8-11.

\vspace{-.3cm} {\em Interfaith Winter Shelter Volunteer} \hfill {\bf
  January 2016 -- Present }\\ Evening shift volunteer for at a
low-barrier winter homeless shelter.


\end{resume}
\end{document}
